%! TEX program = xelatex
\documentclass[a4paper,12pt]{report}

\usepackage{./../packages/mainstyle}
\usepackage{./../packages/boxes}
\usepackage{./../packages/titleENG}

\setcoursename{Very Interesting and Elaborate Course Name}
\setcoursebook{“Extremely Cool and Deep Book” by Famous Author}
\setauthorname{name surname}

\begin{document}

\makefrontpage

\tableofcontents

\chapter{Chapter}
Yes these are notes they are so useful

\section{Introduction}

Yes indeed so true aha i agre yes yes\dots

\subsection{furthermore}
Mmmh terribly interesting indeed \( 5+5=10 \)

\begin{defbox}{group}
A \emph{group} is a set $G$ together with a binary operation 
$\cdot : G \times G \to G$ such that:
\begin{enumerate}
  \item $(a \cdot b) \cdot c = a \cdot (b \cdot c)$ forrr all $a,b,c \in G$ (associativity),
  \item there exists an identity element $e \in G$ such that $e \cdot a = a \cdot e = a$ for all $a \in G$,
  \item every $a \in G$ has an inverse $a^{-1}$ such that $a \cdot a^{-1} = a^{-1} \cdot a = e$.
\end{enumerate}
\end{defbox}

\begin{lemmabox}{Uniqueness of the Identity}
If $e$ and $e'$ are both identity elements in $G$, then $e = e'$.
\end{lemmabox}

\begin{theobox}{Cancellation Law}
In a group $G$, if $a \cdot b = a \cdot c$, then $b = c$.
\end{theobox}

\begin{proofbox}
Suppose $a \cdot b = a \cdot c$. 
Multiply both sides on the left by $a^{-1}$:
\[
a^{-1} \cdot (a \cdot b) = a^{-1} \cdot (a \cdot c).
\]
By associativity,
\[
(a^{-1} \cdot a) \cdot b = (a^{-1} \cdot a) \cdot c.
\]
Since $a^{-1} \cdot a = e$ (the identity), we get $e \cdot b = e \cdot c$,
which simplifies to $b = c$.
\end{proofbox}

\begin{gbox}{stuff}
This is the stuff.
\end{gbox}

\end{document}
