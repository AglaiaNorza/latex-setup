%! TEX program = xelatex
\documentclass[a4paper,12pt]{report}

\usepackage{./../packages/mainstyle}
\usepackage{./../packages/colors}
\usepackage{./../packages/title}
\usepackage{./../packages/packs}
\usepackage{./../packages/macros}

\usepackage{./../packages/frameboxes}
\usepackage{./../packages/noteboxes}
\usepackage{./../packages/roundboxes}

\setcoursename{Course Name}
\setcoursebook{Book Title, \textit{Author Name}}
\setauthorname{Author Name}
\setauthoremail{person@gmail.com}
\setauthorgithub{UserName}

\begin{document}

\tableofcontents

\chapter{Boxes}

\section{Frame Boxes}

The boxes defined in \texttt{frameboxes.sty} are:

\begin{itemize}
        \item \texttt{defframe}
        \begin{defframe}{Title}{}
            This is a definition.
        \end{defframe}

    \item \texttt{cdefframe}
        \begin{cdefframe}{DeepBlueLight}{DeepBlue}{Title}{}
            This definition takes in colors in input. Its counter is shared with \texttt{defframe}.

            The syntax is: \texttt{\textbackslash begin\{cdefframe\}\{bg color\}\{title/side color\}\{title\}}
        \end{cdefframe}

    \item \texttt{cframe}
               \begin{cframe}{BlueGrayLight}{DeepRed}{Custom Title}
                   This is a custom-colored frame. Colors are given in input like with \texttt{cdefframe}, but there is no numbering or ``Def.''
        \end{cframe}

    \item \texttt{lemmaframe}

        \begin{lemmaframe}{The Fundamental Lemma}{label:fund-lemma}

            Is blue, includes ``Lemma <counter>'' (counter is separate from the definitions').

        \end{lemmaframe}

    \item \texttt{thmframe}

        \begin{thmframe}{Pythagorean Theorem}{label:pythag}
            Just like \texttt{lemmaframe}, but green and with a different counter.
        \end{thmframe}

    \item \texttt{propframe}

        \begin{propframe}{Commutativity}{label:commute}
            Just like \texttt{thmframe}, but with a different counter.
        \end{propframe}

    \item \texttt{proofframe}
               
        \begin{pframe}
            Can be used without title... (useful in nested boxes)
        \end{pframe}

        \begin{pframe}[title=Title Example]
            But it takes in one input that can be used to override the \texttt{notitle} option like so:

            \texttt{\\begin\{pframe\}[title=Title Example]}
        \end{pframe}

    \item \texttt{gframe}
        \begin{gframe}{With title}
            General-purpose gray note.
        \end{gframe}

        \begin{gframe}{}
            Title can be left empty (\texttt{\{\}}).
        \end{gframe}

\end{itemize}

\section{Note Boxes}


\section{Round Boxes}



\end{document}
